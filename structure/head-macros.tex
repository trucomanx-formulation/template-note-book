\makeatletter
\newcommand{\FprShowFont}{codifica\c{c}\~ao: \f@encoding{},
  familia: \f@family{},
  serie: \f@series{},
  %shape: \f@shape{},
  e tamanho: \f@size{} pt}
\makeatother

%%%%%%%%%%%%%%%%%%%%%%%%%%%%%%%%%%%%%%%%%%%%%%%%%%%%%%%%%%%%%%%%%%%%%%%%%%%%%%%%
%%%%%%%%%%%%%%%%%%%%%%%%%%%%%%%%%%%%%%%%%%%%%%%%%%%%%%%%%%%%%%%%%%%%%%%%%%%%%%%%

%\usepackage{xcolor}
\usepackage{xkeyval}
\usepackage{shadowtext}


% Configuração das chaves com xkeyval
\makeatletter
% Definindo valores padrão para as chaves offset e color
\newcommand{\TextShadowColor@offsetX}{5pt}
\newcommand{\TextShadowColor@offsetY}{5pt}
\newcommand{\TextShadowColor@color}{red}

\define@key{TextShadowColor}{offsetX}{\renewcommand{\TextShadowColor@offsetX}{#1}}
\define@key{TextShadowColor}{offsetY}{\renewcommand{\TextShadowColor@offsetY}{#1}}
\define@key{TextShadowColor}{color}{\renewcommand{\TextShadowColor@color}{#1}}

\setkeys{TextShadowColor}{offsetX=1pt, offsetY=1pt, color=red}

% Comando personalizado que usa as chaves definidas
\newcommand{\TextShadowColor}[2][]{%
  \setkeys{TextShadowColor}{#1}               % Configura as chaves fornecidas pelo usuário
  \shadowoffsetx{\TextShadowColor@offsetX}       % Usa o valor de offsetX
  \shadowoffsety{\TextShadowColor@offsetY}       % Usa o valor de offsetY
  \shadowcolor{\TextShadowColor@color}         % Usa o valor de color
  \shadowtext{#2}                      % Aplica o texto com contorno
}
\makeatother

%%%%%%%%%%%%%%%%%%%%%%%%%%%%%%%%%%%%%%%%%%%%%%%%%%%%%%%%%%%%%%%%%%%%%%%%%%%%%%%%
%%%%%%%%%%%%%%%%%%%%%%%%%%%%%%%%%%%%%%%%%%%%%%%%%%%%%%%%%%%%%%%%%%%%%%%%%%%%%%%%

%\usepackage{xcolor}
%\usepackage{xkeyval}
\usepackage{contour}


% Configuração das chaves com xkeyval
\makeatletter
% Definindo valores padrão para as chaves offset e color
\newcommand{\TextContourColor@length}{5pt}
\newcommand{\TextContourColor@color}{red}
\newcommand{\TextContourColor@number}{25}

\define@key{TextContourColor}{length}{\renewcommand{\TextContourColor@length}{#1}}
\define@key{TextContourColor}{color}{\renewcommand{\TextContourColor@color}{#1}}
\define@key{TextContourColor}{number}{\renewcommand{\TextContourColor@number}{#1}}

\setkeys{TextContourColor}{length=10pt, color=green, number=25}

% Comando personalizado que usa as chaves definidas
\newcommand{\TextContourColor}[2][]{%
  \setkeys{TextContourColor}{#1}               % Configura as chaves fornecidas pelo usuário
  \contourlength{\TextContourColor@length}       % Usa o valor de length
  \contournumber{\TextContourColor@number}
  \contour{\TextContourColor@color}{#2}                      % Aplica o texto com contorno
}
\makeatother


%%%%%%%%%%%%%%%%%%%%%%%%%%%%%%%%%%%%%%%%%%%%%%%%%%%%%%%%%%%%%%%%%%%%%%%%%%%%%%%%
%%%%%%%%%%%%%%%%%%%%%%%%%%%%%%%%%%%%%%%%%%%%%%%%%%%%%%%%%%%%%%%%%%%%%%%%%%%%%%%%

%-------------------------------------------------------------------------------
% Esquema de colores
%-------------------------------------------------------------------------------

\makeatletter
\newcommand{\DefineBookColorScheme}[6]{
\expandafter\def\csname BookColorScheme@1\endcsname{#1}
\expandafter\def\csname BookColorScheme@2\endcsname{#2}
\expandafter\def\csname BookColorScheme@3\endcsname{#3}
\expandafter\def\csname BookColorScheme@4\endcsname{#4}
\expandafter\def\csname BookColorScheme@5\endcsname{#5}
\expandafter\def\csname BookColorScheme@6\endcsname{#6}
}
\makeatother
